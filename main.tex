\documentclass{article}
\usepackage[utf8]{inputenc}
\usepackage[spanish]{babel}
\usepackage{listings}
\usepackage{graphicx}
\graphicspath{ {images/} }
\usepackage{cite}

\begin{document}

\begin{titlepage}
    \begin{center}
        \vspace*{1cm}
            
        \Huge
        \textbf{Parcial 1}
            
        \vspace{0.5cm}
        \LARGE
         Calistenia
            
        \vspace{1.5cm}
            
        \textbf{Jessica Valentina Gaviria Samboni}
            
        \vfill
            
        \vspace{0.8cm}
            
        \Large
        Despartamento de Ingeniería Electrónica y Telecomunicaciones\\
        Universidad de Antioquia\\
        Medellín\\
        Marzo de 2021
            
    \end{center}
\end{titlepage}

\tableofcontents
\newpage
\section{Instrucciones}\label{intro}
Condiciones:
Realizar los siguientes pasos empleando una sola mano
\newline
1.Tome la hoja de papel y muevala hacia uno de los costados, haciendo que las tarjetas ubicadas abajo queden totalmente descubiertas.
\newline
2.Tome las dos tarjetas juntas, y parelas verticalmente en el centro de la superficie de la hoja.
\newline
3.Mantenga el borde superior de las tarjetas unida, y cuidadosamente separe la parte inferior de las tarjetas, tratando de formar un triangulo con ambas tarjetas, y que se sostengan por si solas.
\section{Conclusiones}\label{intro}
Esta actividad permite reconocer las distintas formas en que una misma situacion, en este caso especifico las instrucciones dadas, pueden ser entendidas de formas diversas, por lo  cual pude reflexionar sobre la importancia de dar instrucciones claras y no ambiguas.
Adicional a lo anterior quiero resaltar lo interesante  de permitir que quienes realizaron la actividad escogieran la forma mas adecuada para cada uno de cumplir con lo instruido en el paso 3 (formar un triangulo con ambas tarjetas), ya que se puede llegar a un mismo punto recorriendo diversos caminos.



\bibliographystyle{IEEEtran}
\bibliography{references}
\cite{classroomInformaticaII}
\end{document}
